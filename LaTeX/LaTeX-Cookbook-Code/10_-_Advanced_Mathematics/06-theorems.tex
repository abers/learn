\documentclass{article}
\usepackage{amsmath}
\usepackage{amsthm}
\newtheorem{thm}{Theorem}
\newtheorem{lem}{Lemma}
\theoremstyle{definition}
\newtheorem{dfn}{Definition}
\theoremstyle{remark}
\newtheorem*{note}{Note}
\begin{document}
\begin{dfn}
  The longest side of a triangle with a right angle
  is called the \emph{hypotenuse}.
\end{dfn}
\begin{note}
  The other sides are called \emph{catheti},
  or \emph{legs}.
\end{note}
\begin{thm}[Pythagoras]
  \label{pythagoras}
  In any right triangle, the square of the hypotenuse
  equals the sum of the squares of the other sides.
\end{thm}
\begin{proof}
  The proof has been given in Euclid's Elements,
  Book 1, Proposition 47. Refer to it for details.
  The converse is also true, see lemma \ref{converse}.
\end{proof}
\begin{lem}
  \label{converse}
  For any three positive numbers \(x\), \(y\),
  and \(z\) with \(x^2 + y^2 = z^2\), there is a
  triangle with side lengths \(x\), \(y\) and \(z\).
  Such triangle has a right angle, and the hypotenuse
  has the length \(z\).
\end{lem}
\begin{note}
  This is the converse of theorem \ref{pythagoras}.
\end{note}
\end{document}
